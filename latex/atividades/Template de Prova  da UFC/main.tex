%%%%%%%%%%%%%%%%%%%%%%%%%%%%%%%%%%%%%%%%%%%%%%%%%%%%%%%%%%%%%%%%%%%%%%%%%%%%%%%%%%
%% This project aims to create a test template or exercise list from the        %%     
%% Federal University of Ceará (UFC).                                           %%
%% author: Maurício Moreira Neto - Doctoral student in Computer Science         %%
%% contacts:                                                                    %%
%%    e-mail: maumneto@ufc.br                                                   %%
%%    linktree:  https://linktr.ee/maumneto                                     %%
%%%%%%%%%%%%%%%%%%%%%%%%%%%%%%%%%%%%%%%%%%%%%%%%%%%%%%%%%%%%%%%%%%%%%%%%%%%%%%%%%%
\documentclass{ufcdocument}

\usepackage[utf8]{inputenc}
\usepackage[portuguese]{babel}

%% Informations that will be insert in the table header 
\def\course{Programação e Algoritmos}
\def\prof{Renato Freitas}
\def\semester{2025.1}
\def\codeCourse{XXXXXX}
\def\registration{}
\def\student{}
\def\graduate{Bacharelado em Ciência da Computação}
\def\theme{Atividade}
\def\thisDoc{atividade}

\begin{document}
    %% Table with the header
    \makeheader
    
    %% Space for the instructions
    \fbox{
        \parbox{\textwidth}{
            \begin{minipage}{\textwidth}
                \makeinstructions
                {
                    \begin{instlist}
                        \item A \thisDoc é individual e não é pesquisada.
                        \item Preencha o cabeçalho da folha pergunta com seus dados.
                        \item  Todas as folhas respostas devem conter o nome a a matrícula do aluno.
                        \item O preenchimento das respostas deve ser feito utilizando caneta (preta ou azul).
                        \item As repostas podem ser em português ou inglês (bônus para quem fizer em inglês)
                    \end{instlist}
                }
            \end{minipage}
        }
    }
    %% Space between the instructions and the questions.
    \vspace{1cm}
    
    \begin{question}
        \item Quais são as duas principais partes que compõem um computador? Dê pelo ao menos três exemplos de cada uma das partes.\points{2}

        \item Escreva com suas palavras o que é algoritmo e o que é um programa.\points{1}
        
        \item Quais as técnicas de elaboração de algoritmo?\points{1}
        
        \item Cite uma desvantagem para cada uma das técnicas de elaboração de um algoritmo.\points{1}
        
        \item Quais as etapas que um programador dever seguir para construir um programa de computador? Discorra sobre cada uma das etapas.\points{2}
    \end{question}

\end{document}