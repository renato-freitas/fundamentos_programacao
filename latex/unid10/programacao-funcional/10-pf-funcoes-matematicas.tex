\section{Funções Metamáticas}

\begin{frame}
	\frametitle{Funções Metamáticas}
	\begin{itemize}
		\item Uma função matemática é um mapeamento de membros de um conjunto, 
		chamado de conjunto domínio, para outro, chamado de conjunto imagem (SEBESTA, 2018).
		\item As funções são geralmente aplicadas a um elemento em particular do conjunto domínio, fornecido como um parâmetro 
		para a função.
		\item Uma função leva a, ou retorna, um elemento do conjunto imagem.
		\item Uma das características fundamentais das funções matemáticas é que a 
		ordem de avaliação de suas expressões de mapeamento é controlada por recursão e expressões condicionais, e não por sequência e repetição iterativa, 
		comuns nas linguagens de programação imperativas.
		\item elas não têm efeitos colaterais, sempre definem o mesmo valor quando fornecido o mesmo conjunto de argumentos.
	\end{itemize}
\end{frame}



% Definição
\begin{frame}
	\frametitle{Funções Metamáticas}
	\begin{itemize}
		\item Funções matemáticas típicas são geralmente escritas como \textbf{um nome de função}, seguido de \textbf{uma lista de parâmetros} entre parênteses, seguidos pela \textbf{expressão de mapeamento}.
	\end{itemize}

	\begin{block}{\textbf{Função Simples}}
		\[ cubo(x) \equiv x * x * x, \mbox{onde } x \mbox{ é um número real } (x \in \mathbb{R}) \]
	\end{block}

	\begin{itemize}
		\item O símbolo $\equiv$ é usado para significar ``é definido como''.
		\item O parâmetro $x$ pode representar qualquer membro do conjunto domínio
	\end{itemize}

	\begin{block}{\textbf{Função Simples Mais Formalemente}}
		\[ cubo: \mathbb{R} \rightarrow \mathbb{R} \]
	\end{block}
\end{frame}


% Aplicação
\begin{frame}
	\frametitle{Funções Metamáticas}
	\begin{itemize}
		\item Aplicações de funções matemáticas são especificadas por um par que contém o \textbf{nome da função} com um \textbf{elemento do conjunto domínio}.
	\end{itemize}

	\begin{block}{\textbf{Aplicação da função $cubo(x) \equiv x * x * x$}}
		\[ cubo(2.0) \mbox{ leva ao valor } 8.0 \]
	\end{block}

	\begin{itemize}
		\item Não há parâmetro desvinculado. Todos são do conjunto domínio da função.
		\item Não há uma variável externa (local ou global)
	\end{itemize}
\end{frame}


\begin{frame}
	\frametitle{Funções Metamáticas}
	\textbf{Função Lambda}
	\begin{itemize}
		\item primeiros trabalhos teóricos acerca de funções separaram a tarefa de defini-las da de nomeá-las.
		\item Notaçao criada po Alonzo Church em 1941.
		\item Método para definir funções sem nome: lista de parâmetros e mapeamento da função.
		\begin{block}{\textbf{Expressão Lambda}}
			\[ \lambda(x)x*x*x \]
		\end{block}

		\begin{block}{\textbf{Aplicando a Expressão Lambda}}
			\[ (\lambda(x)x*x*x)(2) \mbox{ resulta no valor } 8\]
		\end{block}
	\end{itemize}
\end{frame}



\begin{frame}
	\frametitle{Funções de Ordem Superior}
	\begin{itemize}
		\item Uma função de ordem superior, ou forma funcional, é uma que recebe funções 
		como parâmetros ou que leva a uma função como resultado, ou ambos
		\item Composição de Funções
		\begin{block}{\textbf{Forma Funcional}}
			\[ h \equiv f \circ g \]
		\end{block}

		\begin{block}{\textbf{Aplicando a Forma Funcional}}
			\begin{align*} 
				f(x) \equiv 1 + x \\
				g(x) \equiv x^2 \\
				\mbox{ então } h \mbox{ é definida como} \\
				h(x) \equiv f(g(x)) \mbox{ ou } h(x) \equiv 1 + (x^2)
			\end{align*}
		\end{block}
	\end{itemize}
\end{frame}




% APLICAR PARA TODOS
\begin{frame}
	\frametitle{Funções de Ordem Superior}
	\textbf{Aplicar-para-todos}
	\begin{itemize}
		\item Forma funcional que recebe uma 
		única função como um parâmetro. Se aplicada para uma lista de argumentos, 
		aplicar-para-todos aplica seu parâmetro funcional para cada um dos valores 
		no argumento lista e coleta os resultados em uma lista ou em uma sequência. 
		
		\item Denotada por $\alpha$.
		\begin{block}{\textbf{Forma Funcional Aplicar-pata-todos}}
			\[ h(x) \equiv x * x, \mbox{ então} \]
			\[ \alpha (h, (2, 3, 4)) \equiv [h(2), h(3), h(4)]
			\mbox{ resulta em } (4, 9, 16) \]
		\end{block}
	\end{itemize}
\end{frame}