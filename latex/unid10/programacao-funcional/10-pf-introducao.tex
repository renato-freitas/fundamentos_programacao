\def\secUm{Introdução à Programação Funcional}
\section{\secUm}
\begin{frame}
	\frametitle{\secUm}
	Introdução
	\begin{itemize}
		\item Paradigma que Surigiu 1960
		\item Motivada pelo desenvolvimento de IA e subcampos
		\item Atender necessidade não atendida pela programação imperativa
		\item Antiga mas não vingou devido ao pouco poder de processamento à epoca
		\item Baseada em funções matemáticas
	\end{itemize}
\end{frame}




\begin{frame}
	\frametitle{Característica da Programação Funcional}
	\begin{itemize}
		\item Sua computação é vista como uma função matemática mapeando entradas e saídas
		\item Não há a noção de estado
		\item Variávei imutáveis
		\item Não lá laço de repetição
		\item Recursividade
		\item Mais confiável, menos propenso a erros
		\item Sem efeitos colaterais
		\item Linguagens funcionais puras e não-puras ou híbridas
	\end{itemize}
\end{frame}