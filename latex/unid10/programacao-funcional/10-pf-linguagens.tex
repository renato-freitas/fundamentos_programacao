\section{Linguagens de Programação Funcional}

\begin{frame}
	\frametitle{Linguagens de Programação Funcional}
	\begin{itemize}
		\item O objetivo do projeto de uma linguagem de programação funcional é mimetizar funções matemáticas ao máximo possível.
		\item abordagem para a solução de problemas fundamentalmente diferente de abordagens usadas com linguagens imperativas
		\item A execução de uma função sempre produz o 
    mesmo resultado quando fornecidos os mesmos parâmetros. Esse recurso 
    é chamado de \textbf{transparência referencial}.
    \item Há linguagens pura e não-puras ou híbridas.
    \begin{itemize}
      \item Lisp, Scheme, Haskell, F\#, Common Lisp, ML
    \end{itemize}
	\end{itemize}
\end{frame}


