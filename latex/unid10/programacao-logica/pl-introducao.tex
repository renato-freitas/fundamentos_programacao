\section{Introdução à Programação Lógica}

\begin{frame}
	\frametitle{Introdução à Programação Lógica}
	\begin{itemize}
		\item Surgiu em 1970
		\item Baseado em lógica simbólica e processo de inferência
		\item Paradigma Declarativo
		\begin{itemize}
			\item Não procedural. Não detalhar ``como fazer''.
			\item Especifica os resultados esperados ou declara os objetivos da computação (foco em ``o quê'')
		\end{itemize}
		\item Programação que usa lógica simbólica é denominada \textbf{programação lógica}
		\begin{itemize}
			\item linguagem de programação lógica ou linguagem declarativa
		\end{itemize}
		\item Base em lógica formal
	\end{itemize}
\end{frame}


\begin{frame}
	\frametitle{Introdução à Programação Lógica}
	Principais aplicações de programação lógica ou declarativa:
	\begin{itemize}
		\item Inteligência Artifical - Processamento em Linguagem Natural, Racioncínio Automático - Prolog
		\item Banco de Dados - Recuperação de Informações - SQL
	\end{itemize}
\end{frame}


\section{Cálculo de Predicado}
\begin{frame}
	\frametitle{Cálculo de Predicado}
	\begin{itemize}
		\item Programação lógica tem como base a \textbf{lógica formal}.
		\item A lógica formal foi criada para descrever proposições e verificar suas validades.
		\item Uma \textbf{proposição} pode ser vista como uma sentença lógica que pode 
		ou não ser verdadeira e consiste em objetos e relacionamentos entre esses objetos.
		\item a \textbf{lógica simbólica} ou \textbf{lógica de símbolos} é usada para atender a lógica formal em três pontos:
		
		\begin{itemize}
			\item expressar proposições
			\item expressar os relacionamentos entre proposições
			\item descrever como novas proposições são inferidas a partir das anteriores, presumidamente verdadeiras.
		\end{itemize}
	\end{itemize}
\end{frame}



\begin{frame}
	\frametitle{Cálculo de Predicado}
	\begin{itemize}
		\item Há uma forte conexão entra a lógica formal e a matemática 
		\item O conjunto inicial de proposições é formado pelos axiomas da teoria dos números e do conjuntos. 
		\item Teoremais são proposições adicionais que podem ser inferidas a partir do conjunto inicial.
		\item O cálculo de predicados é uma área da lógica matemática que estuda a estrutura e a interpretação de proposições que contêm quantificadores e variáveis, permitindo expressar relações mais complexas entre objetos.
	\end{itemize}

	\begin{block}{oxe}
		``P: Quantas pernas tem um cachorro, se chamarmos sua cauda de perna?
R: Quatro. Chamar uma cauda de perna não a transforma em uma perna.''
		\textbf{Abraham Lincoln}
	\end{block}
\end{frame}