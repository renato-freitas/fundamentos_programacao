\documentclass[11pt]{beamer}
\usepackage{amsmath, amssymb}
\usepackage[utf8]{inputenc}
\usepackage[T1]{fontenc}
\usepackage{lmodern}
\usetheme{Boadilla}

\begin{document}
	\title{Fundamentos de Programação \\ Aula 01}
	\subtitle{Apresentação da Disciplina}
	\author{Renato Freitas}
	\logo{\includegraphics[height=1cm]{D:/logo-ufc.png}}
	\institute[UFC]{Universidade Federal do Ceará - UFC}
	\date{Russas, outubro de 2024}
	\setbeamercovered{transparent}
	\setbeamertemplate{navigation symbols}{}
	\begin{frame}[plain]
		\maketitle
	\end{frame}
	
	
	\begin{frame}
		\frametitle{Programa da Aula}
		\tableofcontents
	\end{frame}

%https://caequfpb.yolasite.com/resources/Livro%20Fundamentos%20da%20Programa%C3%A7%C3%A3o.pdf
\input{aula01-apresentacoes.tex}
\section{Introdução à Programação, algoritmos e Conceitos Fundamentais}
\begin{frame}
	\frametitle{Introdução à Programação, algoritmos e Conceitos Fundamentais}
	\textbf{Objetivo}
	\begin{itemize}
		\item Apresentar os conceitos de \textbf{Programação} e \textbf{Algoritmo}
		\item Discutir as abordagens para as representações de algoritmos
		\item Mostrar os comando básicos
	\end{itemize}
\end{frame}


\begin{frame}
	\frametitle{Programação, algoritmos e Conceitos Fundamentais}
	\textbf{Introdução}
	\begin{itemize}
		\item O ser humano têm criado máquinas para auxiliar em seus trabalhos, incluindo os computadores.
		\item Computadores são:
		\begin{itemize}
			\item extremamente rápidos e precisos.
			\item formado por \textit{hardware} e \textit{software}.
		\end{itemize}
		\item mas são máquinas dependentes.
		\item Precisam de instruções detalhadas para receber, processar, armazenar e retornar dados.
	\end{itemize}
\end{frame}


\begin{frame}
	\frametitle{Programação}
	\textbf{Introdução}
	\begin{itemize}
		\item \textbf{Programação} é o ato de construir um programa de computador (\textit{software}),
		\item Um \textbf{programa} é um conjunto de instruções detalhadas, representadas por um código que o computador entende.
		\item As principais etapas para desenvolver um programa:
		\begin{itemize}
			\item Análise do problema ou tarefa.
			\item Elaboração de um algoritmo.
			\item Codificação.
%			CABE UMA IMAGE AQUI
		\end{itemize}
	\end{itemize}
\end{frame}


\begin{frame}
	\frametitle{Conceito de Algoritmo}
	Algoritmo é uma sequência de passos que visa atingir um objetivo bem definido." (FORBELLONE, 1999)
	
	Algoritmo é a descrição de uma sequência de passos que deve ser seguida para a realização de uma tarefa." (ASCENCIO, 1999)
\end{frame}



\begin{frame}
	\frametitle{Descrição de Algoritmos}
	\textbf{Técnicas para descrição de algoritmos}
	\begin{itemize}
		\item Descrição Narrativa
		\item Fluxograma
		\item Pseudocódigo (que pode ser o Portugol)
	\end{itemize}
\end{frame}



\begin{frame}
	\frametitle{Técnicas para descrição de algoritmos}
	
	\begin{block}{\textbf{Descrição Narrativa}}
		Consiste em analisar o enunciado do problema ou tarefa e escrever os passos para a resolução do problema em linguagem natural (português, inglês etc).
	\end{block}
	
	\textbf{Vantagem:} o conhecimento prévio da linguagem.
	
	\textbf{Desvantagem:} a descrição pode ser ambígua.
	
	\textbf{Exemplo:} ""
\end{frame}

\begin{frame}
	\frametitle{Técnicas para descrição de algoritmos}
	
	\begin{block}{\textbf{Fluxograma}}
		Escrever os passos da solução do problema/tarefa por meio de símbolos gráficos.
	\end{block}
	
	\textbf{Vantagem:} é mais fácil de entender os passos do algoritmo.
	
	\textbf{Desvantagem:} é necessário aprender os significados dos símbolos gráficos e não é detalhado o suficiente.
	
	Exemplo:
\end{frame}


\begin{frame}
	\frametitle{Técnicas para descrição de algoritmos}
	
	\begin{block}{\textbf{Pseudocódigo}}
		Escrever os passos por meio de texto com regras predefinidas ou estruturadas.
	\end{block}
	
	\textbf{Vantagem:} é bem próximo do código final.
	
	\textbf{Desvantagem:} tem que assimilar as regras que devem ser usadas.
\end{frame}




% \begin{frame}
% 	\frametitle{Referências}
% 	\begin{thebibliography}{4}
% 		\bibitem{linaker-2015}[1] Linaker, J., Sulaman, S. M., Host, M., and de Mello, R. M.
% 		\newblock {\em  Guidelines for conducting surveys in software engineering v. 1.1}. 
% 		\newblock Lund University, 50, $2015$.
		
		
% 		\bibitem{abes-2022}[2] ABES.
% 		\newblock {\em Mercado Brasileiro de Software 2022 – Panorama e Tendências}. 
% 		\newblock ABES – Associação Brasileira das Empresas de Software, $2022$.
		
	
% 	\end{thebibliography}
% \end{frame}




\begin{frame}[plain]
	\maketitle
	\centering
	Obrigado!
\end{frame}

\end{document}