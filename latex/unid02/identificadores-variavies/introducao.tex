\section{Constantes e Variáveis}
\section{Identificadores}
\section{Tipos de Dados}
\section{Escopo}

\begin{frame}
	\frametitle{Constantes e Variáveis}
	Em geral, um programa recebe e processa dados. Para o processamento, os dados precisam ser armazenados na memória computador.

	Constantes e Variáveis são posições de memória que possuem nome e tipo.

	o conceito de Constante significa algo que se mantém como está ou como foi definido.\\
	São valores fixos que não podem ser alterados pelas instruções de um algoritmo.

	Exemplos de constantes:\\
	- valor de Pi ($\pi$): $3.14$\\
	- a soma dos ângulos internos de um triângulo $180$º

\end{frame}


\begin{frame}
	\frametitle{Constantes e Variáveis}
	Uma variável pode ter o seu conteúdo alterado durante a execução de um programa.
	Em geral, uma variável só pode armazenar um tipo de valor.\\
	Essa é uma caracteristicas das linguagem fortemente tipadas.\\
	Ou seja quando se determina um tipo a uma variável, o programa não permitirá atribuir outro tipo.\\
	Por exemplo:\\
	Se uma variável $x$ é definida como um texto, não é possível atribuir o número. 

	Um computador possui uma tabela de alocação de memória que contém o nome, o tipo e o endereço inicial da variável.
\end{frame}


\begin{frame}
	\frametitle{Constantes e Variáveis}
	A memória de um computador atua semelhante a um armário com várias gavetas numeradas.\\

	As gavetas são os espaços físicos para armazenas os dados.\\

	As variáveis são diferenciadas por meio de identificadores.\\


	Variáveis simples só armazena um valor por vez, de um tipo.
	Exemplo: x é um inteiro, x := 5, depois x recebe o dobro de 5, x := 10

	Descobri qual gaveta remover o valor antigo e substituir pelo novo.
\end{frame}





\begin{frame}
	\frametitle{Identificadores}
	Identificadores são nomes dados às variáveis, constantes, funções e demais elementos utilizados na programação.

	Buscar um dados na memória basta informar o nome da variável.

	Palavras reservadas???? sÃO Identificadores PREDEFINIDOS para serem usados pelos interpretador
\end{frame}



\begin{frame}
	\frametitle{Identificadores}
	Regras para criar identificadores:\\

	\begin{itemize}
		\item não pode começar com dígito;
		\item pode ter letras, dígitos e \textit{underscore};
		\item não pode conter caracteres especias como: 
		\item não pode ser um palavra reservada de um linguagem de programação
		\item não pode ter espaços em branco. ex "cor vermelha"
	\end{itemize}
	
	Identificadores válidos:
	idade, \_media\_alunos

	inválidos:
	1pwd, nota n2, 
	Declara uma variável:
	template: identificador: tipo\_primitivo
\end{frame}





\begin{frame}
	\frametitle{Palavras reservadas}
	\begin{block}{\textbf{Palavras reservadas em Portugol}}
		inicio senao para enquanto var logico se ate faca inteiro real
	\end{block}
\end{frame}



\begin{frame}
	\frametitle{Tipos de Dados}
	\begin{block}{\textbf{Tipos Primitivos}}
		\begin{itemize}
			\item \textbf{Numérico}
			\item \textbf{Lógico}
			\item \textbf{Literal ou Caractere}:
		\end{itemize}
	\end{block}
\end{frame}




\begin{frame}
	\frametitle{Tipos de Dados}
		\begin{itemize}
			\item \textbf{Numérico}: os dados númericos se dividem em dois grupos: \textit{inteiros} e \textit{reais}
			\item podem ser positivos ou negativos
			\item não possuem casas decimais
			\item ocupa 2 bytes na memória
		\end{itemize}
\end{frame}




\begin{frame}
	\frametitle{Tipos de Dados}
		\begin{itemize}
			\item \textbf{Lógicos}: também conhecidos com "booleanos".
			\item Só assumem dois valores: \textit{verdadeiro} ou \textit{falso}
		\end{itemize}
\end{frame}



\begin{frame}
	\frametitle{Tipos de Dados}
		\begin{itemize}
			\item \textbf{Literal ou Caractere}: define variáveis que armazenarão um texto
			\item  Esse textos são um formados por um único caratere ou por uma cadeia de caracteres. 
		\end{itemize}
\end{frame}







\begin{frame}
	\frametitle{Exercício}
	Qual(is) o(s) tipo(s) primitivo de dados que aparecem nas sentenças a seguir:
	\begin{itemize}
		\item O título do livro é "A menina que roubava livros".
		\item Este prédio terá vinte andares e custará R\$ 250.300,00 reais cada apartamento.
		\item O carro dele percorre 100 metros em 11.34 segundos.
		\item Você é brasileiro?
		\item A primeira letra do alfabeto é "a". Isso está correto?
	\end{itemize}
\end{frame}