\section{Introdução à Lógica de Programação}


\begin{frame}
	\frametitle{Introdução à Lógica de Programação}
	\textbf{Objetivos}
	\begin{itemize}
		\item Apresentar os conceitos elementares de lógica e sua aplicação no cotidiano. 
		
		\item Definir algoritmo. 
		
		\item Estabelecer uma relação entre lógica e algoritmos: a lógica de programação. 
		
		\item Exemplificar a aplicação dos 
		algoritmos utilizando situações do dia-a-dia. 
		
		\item Comparar as principais formas de representação dos algoritmos.
	\end{itemize}
\end{frame}



\begin{frame}
	\frametitle{Introdução à Lógica de Programação}
	\textbf{O que é Lógica?}
	\begin{itemize}
		\item  corriqueiro da palavra lógica está normalmente relacionado à coerência e à racionalidade
		
		\item normalmente associado à matemática (e as outras ciências?)
		
		\item Podemos relacionar a lógica com a ‘correção do pensamento’, pois uma de suas preocupações é determinar quais operações são válidas e quais não são, fazendo análises das formas e leis do pensamento. 
		
		\item  Como filosofia, ela procura saber por que pensamos assim não de outro jeito. Com arte ou técnica, ela nos ensina a usar corretamente as leis do pensamento.
	\end{itemize}
\end{frame}



\begin{frame}
	\frametitle{Introdução à Lógica de Programação}
	\textbf{O que é Lógica?}
	\begin{itemize}
		\item Lógica matemática: correção do pensamento (raciocínio)
		\item Distinguir o raciocínio correto do incorreto
		\item Questão Central: A conclusão que se obteve é derivada das premissas pressupostas?
		
		Exemplo: ``Todos os homens são mortais.``\\Sócrates é um homem.''\\``Logo, Sócrates é mortal.''
		
		\item Silogismo: estudo da Lógica Proposicional (ou Cálculo Sentenciai) representam um argumento composto de duas premissas e uma conclusão; e está estabelecendo uma relação, que pode ser válida ou não.
		
		\item Esse é um dos objetivos da lógica, o estudo de técnicas de formalização, dedução e análise que permitam verificar a validade de argumentos. 
		
		\item a lógica também objetiva a criação de uma representação mais formal, que se contrapõe à linguagem natural, que é suscetível a argumentações informais (ambiguidade).
	\end{itemize}
\end{frame}




\begin{frame}
	\frametitle{Lógica de Programação}
	Introdução
	\begin{itemize}
		\item Lógica na Filosofia: a arte de pensar corretamente
		\item Porque pensamos dessa forma e não de outra?
		\item sofisma: um raciocínio errado que tenta passar com verdadeiro.
	\end{itemize}
\end{frame}




\begin{frame}
	\frametitle{Introdução à Lógica de Programação}
	\textbf{Lógica no dia-dia}
	\begin{itemize}
		\item quando queremos pensar, falar, escrever ou agir corretamente, precisamos colocar ‘ordem no pensamento’, isto é, utilizar lógica.
	\end{itemize}
	
	\begin{block}{\textbf{Exemplo}}
		\begin{itemize}
			\item A gaveta está fechada.\\
			
			A caneta está dentro da gaveta.\\
			
			Precisamos primeiro abrir a gaveta para depois pegar a caneta.
		\end{itemize}
	\end{block}
\end{frame}











\begin{frame}
	\frametitle{Introduçao à Lógica de Programação}
	\textbf{Lógica de Programação}
	\begin{itemize}
		\item Significa o uso correto das leis do pensamento, da ‘ordem da razão’ e de processos de raciocínio e simbolização formais na programação de computadores, objetivando a racionalidade e o desenvolvimento de técnicas que cooperem para a produção de soluções logicamente válidas e coerentes, que resolvam com qualidade os problemas que se deseja programar.
		
		\item O raciocínio é algo abstrato, intangível. Os seres humanos têm a capacidade de expressálo através da palavra falada ou escrita, que por sua vez se baseia em um determinado idioma, que segue uma série de padrões (gramática). Um mesmo raciocínio pode ser expresso em qualquer um dos inúmeros idiomas existentes, mas continuará representando o mesmo raciocínio, usando apenas outra convenção.
		
		\item Algo similar ocorre com a Lógica de Programação, que pode ser concebida pela mente treinada e pode ser representada em qualquer uma das inúmeras linguagens de programação existentes. Essas, por sua vez, são muito atreladas a uma grande diversidade de detalhes computacionais, que pouco têm a ver com o raciocínio original. Para escapar dessa torre de Babel e, ao mesmo tempo, representar mais fielmente o raciocínio da Lógica de Programação, utilizamos os Algoritmos.
	\end{itemize}
\end{frame}




\begin{frame}
	\frametitle{Introduçao à Lógica de Programação}
	\textbf{Algoritmizando a Lógica}
	\begin{itemize}
		\item O objetivo principal do estudo da Lógica de Programação é a construção de algoritmos coerentes e válidos. Mas o que é um algoritmo?
		
		\item Um algoritmo pode ser definido como uma seqüência de passos que visam a atingir um objetivo bem definido (Forbellone, 2009).
		
		\item Um algoritmo tem por objetivo representar mais fielmente o raciocínio envolvido na Lógica de Programação e, dessa forma, permite-nos abstrair de uma série de detalhes computacionais, que podem ser acrescentados mais tarde. Assim, podemos focalizar nossa atenção naquilo que é importante: a lógica da construção de algoritmos.
	\end{itemize}
\end{frame}