\section{Apresentações}

\begin{frame}
	\frametitle{Apresentações}

	\textbf{O Professor}
	\begin{itemize}
		\item \textbf{Nome:} José Renato da Silva Freitas
		\item \textbf{Formação:} Bacharel e Mestre em Ciência da Computação (UFC), com doutorado em andamento.
		\item \textbf{Linha de Pesquisa:} Banco de Dados \& Sistema de Informação, com ênfase em Integração Semântica e Grafos de Conhecimento.
	\end{itemize}
\end{frame}


\begin{frame}
	\frametitle{Apresentações}

	\textbf{O Estudante}
	\begin{itemize}
		\item Qual o seu nome?
		\item De onde você é?
		\item O que espera do curso?
		\item Já tem experiência com programação?
	\end{itemize}
\end{frame}






\section{Ementa}
\begin{frame}
	\frametitle{Ementa}

	\textbf{Ementa:}
	\begin{itemize}
		\item Algoritmos, Conceitos Fundamentais de Programação
		\item Expressões, Controles de Fluxo
		\item Funções e Procedimentos
		\item Vetores e Matrizes, Cadeias de Caracteres
		\item Tipos Estruturados e Arquivos
	\end{itemize}
\end{frame}


\section{Bibliografia}
\begin{frame}
	\frametitle{Bibliografia}

	\textbf{Bibliografia Básica:}
	\begin{itemize}
		\item 1. MEDINA, M.; FERTIG, C. Algoritmos e programação: teoria e prática 2ed. Novatec, 2004.ISBN:
		9788575220733/857522073X.
		\item 2. ASCENCIO, A. F. G.;CAMPOS, E. A. V. Fundamentos da programação de computadores: algoritmos,
		Pascal, C/C++ e Java. 2 ed. Prentice Hall, 2007. ISBN: 978576051480.
		\item 3. CELES, W.; CERQUEIRA, R.; RANGEL, J. L. Introdução à estrutura de dados: com técnica de
		programação em C. Elsevier, 2004. ISBN: 8535212280
	\end{itemize}
\end{frame}


\begin{frame}
	\frametitle{Bibliografia}
	
	\textbf{Bibliografia Complementar:}
	\begin{itemize}
		\item FORBELLONE, A. L. V. ; EBERSPACHER, H.F. Lógica de programação: a construção de algoritmos. 3
		ed. Prentice Hall, 2005.
		\item Fundamentos de Programação - 3ª Ed. Joyanes, Luis Aguilar; Joyanes, Luis Aguilar. Amgh Editora
		\item Fundamentos De Programação Usando C - 4ª Ed. De Sá, Marques, Lidel – Zamboni.
		\item Lógica de Programação - 3ª Edição. Forbellone, André L. V. Makron Books.
		\item Algoritmos - Lógica para Desenvolvimento de Programação de Computadores. Oliveira, Jayr Figueiredo de; Manzano, José Augusto N. G.. Editora Érica
	\end{itemize}
\end{frame}




\section{Metodologia \& Avaliação}
\begin{frame}
	\frametitle{Metodologia \& Avaliação}
	
	\textbf{Métodos de aula:}
	\begin{itemize}
		\item Aulas expositivas presenciais
		\item Aulas práticas (sala de aula e laboratório)
		\item Trabalhos
	\end{itemize}
	
	\vspace{10pt}

	\textbf{Metodos de avaliação:}
	\begin{itemize}
		\item Duas provas (60\%)
		\item Atividades práticas em sala de aula (40\%)
	\end{itemize}

	\vspace{10pt}
	Discentes com \(4.0 \le \mbox{média} < 7.0\) farão avaliação final no fim do semestre.
\end{frame}